\documentclass[a4paper,11pt]{article}
% \usepackage{fullpage}
% \usepackage[top=2cm, bottom=2cm, left=2cm, right=2cm]{geometry}
\usepackage[margin=2cm]{geometry}
\usepackage[charter]{mathdesign}
\usepackage{epsfig}
\usepackage{graphicx}
% \usepackage{amssymb}
% \usepackage{amsmath}
\usepackage{verbatim}
\usepackage{fancyhdr}
\usepackage{pifont}% http://ctan.org/pkg/pifont
\usepackage[rgb]{xcolor}
\usepackage{tikz}
\usetikzlibrary{matrix,positioning,fit,shapes,arrows,shadows,calc,backgrounds}
\usepackage{pgfplots}
\usepackage{natbib}
\usepackage{enumitem}
\usepackage{pgfgantt}
\usepackage{booktabs}
\usepackage{url}
\usepackage{xspace}
\usepackage{wrapfig}
\usepackage[show]{chato-notes}
\usepackage{multirow}
\usepackage[official]{eurosym}
\usepackage[bottom]{footmisc}

\usepackage{lipsum}% http://ctan.org/pkg/{graphicx,lipsum}
\newcommand{\PRLsep}{\noindent\makebox[\linewidth]{\resizebox{0.3333\linewidth}{1pt}{$\bullet$}}\bigskip}

\definecolor{verylightmagenta}{rgb}{0.95,0.96,1.0}
\definecolor{brightred}{rgb}{0.90,0.2,0.2}

\newcommand{\ag}[1]{\vspace{1mm}\noindent{\color{orange}{\textbf{Comment:} #1}}}
% \newcommand{\ag}[1]{}
\newcommand{\instructions}[1]{\vspace{1mm}\noindent{\color{blue}{#1}}}
% \newcommand{\instructions}[1]{}

\newcommand{\mpara}[1]{\medskip\noindent{\bf #1}}
\newcommand{\spara}[1]{\smallskip\noindent{\bf #1}}
\newcommand{\acronym}{{\sf\small DIFIAS}\xspace}
\newcommand{\acronymtitle}{{\sf\Large DIFIAS}\xspace}

%\newcommand{\proposaltitle}{{Combating bias and polarization in online media}}
%\newcommand{\proposaltitle}{{Balancing content and reducing controversy in social media}}
\newcommand{\proposaltitle}{{Diversity and fairness in information-access systems}}
\newcommand{\proposalabstitle}{{Diversity and fairness in information-access systems}}

\newcommand{\NP}{{\ensuremath{\mathbf{NP}}}}

% CoG problems
\newcommand{\structure}{{\sc Structure}\xspace}
\newcommand{\dense}{{\sc Dense}\xspace}
\newcommand{\important}{{\sc Important}\xspace}
\newcommand{\event}{{\sc Event}\xspace}
\newcommand{\summarize}{{\sc Summarize}\xspace}

% AdG problems
\newcommand{\model}{{\sc\large Modeling}\xspace}
\newcommand{\discover}{{\sc\large Discovery}\xspace}
\newcommand{\explore}{{\sc\large Exploration}\xspace}
\newcommand{\recommend}{{\sc\large Recommendation}\xspace}
\newcommand{\ui}{{\sc\large UserInterface}\xspace}
\newcommand{\methods}{{\sc\large methods}\xspace}

\newcommand{\rto}{{\bf RT1}\xspace}
\newcommand{\rtw}{{\bf RT2}\xspace}
\newcommand{\rtr}{{\bf RT3}\xspace}

%% squishlist
\newcommand{\squishlist}{\begin{list}{$\bullet$}
  { \setlength{\itemsep}{-1pt}
     \setlength{\parsep}{2pt}
     \setlength{\topsep}{2pt}
     \setlength{\partopsep}{0pt}
     \setlength{\leftmargin}{1.5em}
     \setlength{\labelwidth}{1em}
     \setlength{\labelsep}{0.5em} } }
\newcommand{\squishend}{
\end{list}  }


%% gantt stuff

% \definecolor{foobarblue}{RGB}{0,153,255}
% \definecolor{foobaryellow}{RGB}{234,187,0}
% \newganttchartelement{foobar}{
%     foobar/.style={
%         shape=rounded rectangle,
%         inner sep=0pt,
%         draw=foobarblue!50!black,
%         very thick,
%         top color=white,
%         bottom color=foobarblue!50
%     },
%     foobar incomplete/.style={
%         /pgfgantt/foobar,
%         draw=foobaryellow,
%         bottom color=foobaryellow!50
%     },
%     foobar label font=\slshape,
%     foobar left shift=-.1,
%     foobar right shift=.1
% }

%% tikz stuff

\pgfdeclarelayer{background}
\pgfdeclarelayer{foreground}
\pgfsetlayers{background,main,foreground}


\makeatletter
\tikzset{multicircle/.style  args={#1, #2}{%
 alias=tmp@name, % 
  postaction={%
    insert path={
     \pgfextra{% 
     \pgfpointdiff{\pgfpointanchor{\pgf@node@name}{center}}%
                  {\pgfpointanchor{\pgf@node@name}{east}}%            
     \pgfmathsetmacro\insiderad{\pgf@x}%
     %\foreach \c [count=\ci from = 0, evaluate=\ci as \angle using 360 - (\ci) * #1] in {#2}%
        \fill[white] (\pgf@node@name.center)  circle (\insiderad-\pgflinewidth);%
        \draw[#2] (\pgf@node@name.center)  circle (\insiderad-\pgflinewidth);%
        \fill[#2] (\pgf@node@name.center)  -- ++(0:\insiderad-\pgflinewidth) arc (0:#1:\insiderad-\pgflinewidth)--cycle;%
        }}}}}
\makeatother

\definecolor{yafaxiscolor}{rgb}{0.3, 0.3, 0.3}
\definecolor{yafcolor1}{rgb}{0.4, 0.165, 0.553}
\definecolor{yafcolor2}{rgb}{0.949, 0.482, 0.216}
\definecolor{yafcolor3}{rgb}{0.47, 0.549, 0.306}
\definecolor{yafcolor4}{rgb}{0.925, 0.165, 0.224}
\definecolor{yafcolor5}{rgb}{0.141, 0.345, 0.643}
\definecolor{yafcolor6}{rgb}{0.965, 0.933, 0.267}
\definecolor{yafcolor7}{rgb}{0.627, 0.118, 0.165}
\definecolor{yafcolor8}{rgb}{0.878, 0.475, 0.686}
\definecolor{yafcolor9}{rgb}{0.965, 0.733, 0.767}

\newlength{\yafaxispad}
\setlength{\yafaxispad}{-4pt}
\newlength{\yaftlpad}
\setlength{\yaftlpad}{\yafaxispad}
\addtolength{\yaftlpad}{-0pt}
\newlength{\yaflabelpad}
\setlength{\yaflabelpad}{-2pt}
\newlength{\yafaxiswidth}
\setlength{\yafaxiswidth}{1.2pt}
\newlength{\yafticklen}
\setlength{\yafticklen}{2pt}

\makeatletter
\def\pgfplots@drawtickgridlines@INSTALLCLIP@onorientedsurf#1{}
\makeatother

\newcommand{\yafdrawxaxis}[2]{
	\pgfplotstransformcoordinatex{#1}\let\xmincoord=\pgfmathresult 
	\pgfplotstransformcoordinatex{#2}\let\xmaxcoord=\pgfmathresult 
	\pgfsetlinewidth{\yafaxiswidth} 
	\pgfsetcolor{yafaxiscolor}
	\pgfpathmoveto{\pgfpointadd{\pgfpointadd{\pgfplotspointrelaxisxy{0}{0}}{\pgfqpointxy{\xmincoord}{0}}}{\pgfqpoint{-0.5\yafaxiswidth}{\yafaxispad}}}
	\pgfpathlineto{\pgfpointadd{\pgfpointadd{\pgfplotspointrelaxisxy{0}{0}}{\pgfqpointxy{\xmaxcoord}{0}}}{\pgfqpoint{0.5\yafaxiswidth}{\yafaxispad}}}
	\pgfusepath{stroke}

}
\newcommand{\yafdrawyaxis}[2]{
	\pgfplotstransformcoordinatey{#1}\let\ymincoord=\pgfmathresult 
	\pgfplotstransformcoordinatey{#2}\let\ymaxcoord=\pgfmathresult 
	\pgfsetlinewidth{\yafaxiswidth} 
	\pgfsetcolor{yafaxiscolor}
	\pgfpathmoveto{\pgfpointadd{\pgfpointadd{\pgfplotspointrelaxisxy{0}{0}}{\pgfqpointxy{0}{\ymincoord}}}{\pgfqpoint{\yafaxispad}{-0.5\yafaxiswidth}}}
	\pgfpathlineto{\pgfpointadd{\pgfpointadd{\pgfplotspointrelaxisxy{0}{0}}{\pgfqpointxy{0}{\ymaxcoord}}}{\pgfqpoint{\yafaxispad}{0.5\yafaxiswidth}}}
	\pgfusepath{stroke}
}

\newcommand{\yafdrawaxis}[4]{\yafdrawxaxis{#1}{#2}\yafdrawyaxis{#3}{#4}}

\pgfplotscreateplotcyclelist{yaf}{% 
{yafcolor1,mark options={scale=0.75},mark=o}, 
{yafcolor2,mark options={scale=0.75},mark=square},
{yafcolor3,mark options={scale=0.75},mark=triangle},
{yafcolor4,mark options={scale=0.75},mark=o},
{yafcolor5,mark options={scale=0.75},mark=o},
{yafcolor6,mark options={scale=0.75},mark=o},
{yafcolor7,mark options={scale=0.75},mark=o},
{yafcolor8,mark options={scale=0.75},mark=o}} 

\pgfplotsset{axis y line=left, axis x line=bottom,
	tick align=outside,
	compat = 1.3,
	tickwidth=\yafticklen,
	clip = false,
	every axis title shift = 0pt,
    x axis line style= {-, line width = 0pt, opacity = 0},
    y axis line style= {-, line width = 0pt, opacity = 0},
    x tick style= {line width = \yafaxiswidth, color=yafaxiscolor, yshift = \yafaxispad},
    y tick style= {line width = \yafaxiswidth, color=yafaxiscolor, xshift = \yafaxispad},
    x tick label style = {font=\scriptsize, yshift = \yaftlpad},
    y tick label style = {font=\scriptsize, xshift = \yaftlpad},
    every axis y label/.style = {at = {(ticklabel cs:0.5)}, rotate=90, anchor=center, font=\scriptsize, yshift = -\yaflabelpad},
    every axis x label/.style = {at = {(ticklabel cs:0.5)}, anchor=center, font=\scriptsize, yshift = \yaflabelpad},
    x tick label style = {font=\scriptsize, yshift = 1pt},
    grid = major,
    major grid style  = {dash pattern = on 1pt off 3 pt},
	every axis plot post/.append style= {line width=\yafaxiswidth} ,
	legend cell align = left,
	legend style = {inner sep = 1pt, cells = {font=\scriptsize}},
	legend image code/.code={%
		\draw[mark repeat=2,mark phase=2,#1] 
		plot coordinates { (0cm,0cm) (0.15cm,0cm) (0.3cm,0cm) };% 
	} 
}



\pagestyle{fancy}
\lhead[{Gionis}]{{Gionis}}
\rhead[{\acronym}]{{\acronym}}
\chead[~]{{~}}
% \setcounter{page}{1}

\renewcommand{\baselinestretch}{1.0} 
\begin{document}


\begin{center} 
% {\large Vetenskapsrådet: Distinguished professor grant within natural and engineering sciences 2024} \vspace{2.5mm}\\
{\Large Research plan} \vspace{3mm}\\
{\Large\bf {\proposaltitle} {\sc (}{\acronymtitle}{\sc )}}  \vspace{3mm} \\
{\Large Aristides Gionis} 
\end{center}

\subsection*{1.~~~Purpose and aims}

%%%\vspace{-2mm}
In the modern information age we are submersed in a constant stream of data from various sources, 
ranging from the internet and social media to news outlets and podcasts. 
While this wealth of information has the potential to empower individuals, 
it also poses a significant challenge known as \emph{information~overload}:
as the volume of available information continues to grow exponentially, 
individuals are increasingly struggling to sift through noisy data
to find reliable, relevant, and meaningful~content.

\iffalse
This overload can lead to decision paralysis, reduced productivity, and cognitive overload, ultimately hindering our ability to make informed decisions and engage critically with the world around us. Addressing the problem of information overload requires not only technological solutions such as improved search algorithms and filtering mechanisms but also a concerted effort to cultivate digital literacy skills and promote mindful consumption habits among individuals.
\fi 

Mitigating the problem of information overload, 
in addition to cultivating mindful consumption habits, 
requires \emph{technological solutions} tailored to the needs of individuals. 
Advanced \emph{information-retrieval algorithms} can help streamline search processes, 
enabling users to find relevant content. 
%  more efficiently amidst vast data pools. 
\emph{Recommender systems} play a crucial role by suggesting personalized content 
based on users' preferences and past interactions. 
\iffalse
The challenge of information overload arises in various scenarios. 
For instance, it occurs when users interact with search engines using keyword searches, 
when peruse product reviews, 
scroll through social-media timelines, or 
receive recommendations for movies or restaurants.
\fi

%  thus reducing the burden of sifting through irrelevant information. 
% Personalization further enhances user experiences by tailoring content delivery to individual interests and behaviors, ensuring that users receive the most pertinent information. 
% By harnessing these technological tools, we can navigate the information landscape more effectively, mitigating the challenges posed by information overload and empowering individuals to make informed decisions.

When designing smart tools for information access, ensuring \emph{relevance} is paramount:
users expect~content that aligns closely to their interests and needs.
Beyond relevance, however, \emph{diversity} and \emph{fairness} are equally crucial.
Diversity ensures exposure to varied perspectives, 
while fairness ensures equitable access to information.
Integrating diversity and fairness into information-access systems has numerous benefits, 
such as mitigating biases, 
helping counter filter-bubble and rabbit-hole effects, 
fostering a more inclusive and fair representation of different perspectives, and 
promoting a well-rounded understanding of topics.

While relevance, diversity, and fairness
have been extensively studied in different areas in computer science, 
% including information retrieval, recommender systems, and knowledge discovery,
their interplay remains relatively understudied.
We posit that incorporating diversity and fairness into modern information-access systems
remains an unsolved problem, 
offering ample space for fundamental research contributions.
The \acronym\ project aspires to address
significant challenges in this area. 
% More concretely, the project has the following goal.

\medskip
\noindent
\hspace{-3mm}\colorbox{verylightmagenta}{
\begin{minipage}{\textwidth}
{\bf High-level goal of \acronym:} 
We will develop theoretical foundations and novel abstractions to 
study notions of diversity and fairness in information-access systems.
We will design algorithms for these problems with provable guarantees.
Among different information-access systems we will focus on 
information-exploration systems, information networks, and two-sided information markets.
\end{minipage}}

% \ag{Text below could also go to ``novelty''.}

\iffalse
\medskip
\noindent
\hspace{-3mm}\colorbox{verylightmagenta}{
\begin{minipage}{\textwidth}
{\bf Hypothesis:} 
We postulate that modern information-access systems suffer from lack of diversity 
and unfair representation of content. 
We hypothesize that such deficiencies can be mitigated by formulating novel abstractions
and designing rigorous computational methods to support individuals in 
maximizing diversity and improving fairness of the available content. 
\end{minipage}}
\fi

\mpara{Objectives.}
Our overarching objective is to address deficiencies of modern information-access systems~with 
regard to lack of diversity and unfair representation of content.
To achieve this objective we aim to consolidate existing approaches, 
including our recent and ongoing work,  
and push the state-of-the-art 
by intro\-duc\-ing novel abstractions, 
developing rigorous computational methods, and 
performing evaluations on real-world applications.
In particular, {\acronym} has the following research objectives. 
%
\begin{description}
\setlength{\itemsep}{-4pt}
\item[{Models and problems:}]
Develop novel models and novel problem formulations that enable 
obtaining a deeper understanding on phenomena related 
to lack of diversity and fairness in  modern information-access systems.
Focus on three specific domains: 
\emph{information-exploration systems}, \emph{information networks}, and 
\emph{two-sided information markets}.

\item[{Algorithms:}]
Develop computational methods for the formulated problems.
Our methods will be designed for different computational settings, 
e.g., combinatorial formulations, stochastic and uncertain data, 
reinforcement learning, algorithms with predictions, and more.
The proposed algorithms should be efficient %, 
% should be able to deal with uncertainty, 
and should offer theoretical guarantees.

% \item[\manet\ {Limitations:}] Study the proposed structure-discovery problems under the different computational models in order to understand their fundamental limitations, and  develop hardness results or lower bounds.

\item[{Applications:}]
Apply the developed methodology on different application scenarios 
and evaluate the resulting algorithms on real-world benchmark datasets.
% Validate proof-of-concept by showcasing findings of the methods on different use cases. 
Implement the developed algorithms and make them available to the scientific community.

\item[{Research environment in KTH:}]
Strengthen the area of algorithmic data analysis 
at the department of computer science in KTH. 
Nurture doctoral students and postdoctoral researchers in the topic of the project,
and create synergies with other faculty working on 
the foundations of data science, machine learning, and artificial intelligence.
\end{description}

\spara{Diversity vs.\ fairness.}
Diversity maximization and fairness assurance are both important considerations in machine learning.
While they share some common goals, such as promoting inclusivity and reducing discrimination, 
they address different dimensions and require distinct methodologies and techniques. 

Diversity maximization often relies on optimizing distance-based or coverage-based objectives, 
designed to ensure that the outputs generated by a system cover a wide range of perspectives. 
Diversity maxi\-mi\-za\-tion is often applied in recommendation systems, search engines, 
and other decision-making systems. 
% where providing a diverse set of options or perspectives is desirable.
%
On the other hand, 
fairness aims to mitigate biases and ensure equitable treatment of individuals or groups.
% Fairness methods are based on ensuring or optimizing fairness criteria, or post-processing techniques. 
Fairness is often studied in contexts where decisions may impact individuals or groups differently, 
and finds applications in domains such as hiring, criminal justice, and healthcare.

In this project we consider that fairness concepts can be extended to other contexts,  
beyond individuals and demographic groups. 
For instance, we can talk about fairness in the context of a news recommender system, 
asking to ensure that a set of news articles represent fairly the political spectrum.
In this way, similar to diversity, we can study fairness in the context of information-access systems.
Furthermore, note that diversity and fairness are not substitute for one another, 
but they are complementary concepts.

\subsection*{2.~~~State of the art}

%%%\vspace{-2mm}
Maximizing diversity and promoting fairness have been studied 
extensively in different contexts in computer science, 
especially in recent times with the raising importance of \emph{responsible AI}~\cite{dignum2019responsible}.
Due to space limitations we only discuss some representative approaches, 
and their connection with this proposal.

% \spara{diversity as an optimization problem}

From the theoretical viewpoint, the literature mainly focuses on two notions of diversity: 
\emph{coverage-based diversity}, relying on \emph{submodular} coverage functions~\cite{bach2013learning} 
and \emph{pairwise dissimilarity-based} diversity, like \emph{dispersion}~\cite{hassin1997approximation}.
Diversity has also been studied in an \emph{axiomatic} framework~\cite{gollapudi2009axiomatic}. 
An appealing formulation is the \emph{max-sum diversity} problem~\cite{borodin2012max},
which captures trade-offs between diversity and relevance.
Solutions for max-sum diversity involve \emph{combinatorial methods}~\cite{borodin2012max} or
\emph{convex programming}~\cite{cevallos2016max}, 
% \emph{local search}~\cite{cevallos2019improved}, 
among other.
Such formulations and techniques will form a basis for {\acronym}
to study extensions and improved methods. 

% \spara{diversity in recommender systems}

Diversity has also been studied in \emph{recommender systems},
where it has been shown to improve user experience~\cite{Castells2022}.
% and thus, it has received considerable attention recently. 
% ~\cite{on_unexpectedness, improving_aggregate, evaluating_novel_recs, diversity_top_n_recs, rank_relevance}.
One classic method in \emph{information retrieval}
% seeking to strike a balance between diversity and relevance, 
is the \emph{maximal marginal relevance} (MMR)~\cite{MMR}, 
while many other strategies seek to 
maximize some utility function that combines relevance and diversity~\cite{DUM,DPMF}.
% based on submodular functions~\cite{DUM}, probabilistic matrix factorization~\cite{DPMF},  and other.
In this project, we aim to extend the state of the art 
by incorporating models of \emph{user behavior}, 
investigating approaches based on \emph{reinforcement learning} and \emph{algorithms with predictions}, 
and studying novel notions for \emph{fair representation} of recommended items. 

% \spara{diversity in graphs}

Diversity and fairness in graph settings is significantly less-studied area,
compared to item-selection and recommender-systems problems.
Still several ideas have been investigated, 
such as \emph{adding} or \emph{rewiring graph edges} for
\emph{reducing polarization}~\cite{adriaens2023minimizing,cinus2023rebalancing,haddadan2022reducing},
\emph{mitigating exposure to harmful content}~\cite{coupette2023reducing,fabbri2022rewiring}, 
or \emph{improving fairness for the Page\-Rank algorithm}~\cite{tsioutsiouliklis2022link}.
Extending this line of work and introducing new models and methods for graph settings
is a central goal of \acronym.
In addition, while there is a growing amount of literature for 
recommendations in two-sided information markets,
addressing also fairness issues~\cite{do2021two,wang2021user}, 
the area is still not well explored and there is lack of a unified theory
and established methods.

% \spara{fairness in ML}

The topic of bias and fairness in machine learning has received a lot of interest in the recent years, 
having spawn a large community and dedicated dissemination venues, such as the FAccT Conference. 
While many surveys and tutorials can be found online~\cite{caton2020fairness,mehrabi2021survey}, 
we note that our project is more closely related to notions of
\emph{bias and fairness in unsupervised machine learning}, 
such as 
\emph{fair clustering}~\cite{chierichetti2017fair}
% \emph{diversity-aware clustering}~\cite{thejaswi2021diversity}, 
and \emph{fair graph mining}~\cite{dong2023fairness}.
%
The PI, together with his team and research collaborators, 
have pioneered work on many of the above topics,
including
different formulations for diversity in network analysis~\cite{adriaens2023minimizing,cinus2023rebalancing,coupette2023reducing,oettershagen2024finding}
diversity in ranking problems~\cite{zhang2022ranking}, 
as well as for diversity-aware clustering~\cite{thejaswi2021diversity}.

\subsection*{3.~~~Significance and scientific novelty}

\instructions{
Describe briefly how the project relates to previous research within the area, 
and the impact the project may have in the short and long term. 
Describe also how the project moves forward or innovates the current research frontier.}

\ag{
Maybe some comments from state of the art can be moved here, or rephrased. \\
Also say the main novelty would be to study diversity and fairness in new settings.
}

%%%\vspace{-2mm}
\para{Significance.}
Our project lies in the intersection of different subjects in computer-science:
information retrieval, recommender systems, knowledge discovery, and algorithms design. 
While many of the themes have been studied in different contexts,
% Diversity in information access ensures that visible content 
% encompasses a wide range of perspectives, cultures, and voices.
% Additionally, fairness ensures that provided information is equitable and unbiased.
in this project we will push forward the state of the art 
by studying diversity and fairness in new settings.
Our methods will contribute to creating platforms that foster a more equitable and informed society, 
and mitigate the propagation of harmful biases online.

\spara{Scientific novelty.}
The project will make contributions in several different directions.
First, we will design novel frameworks that model faithfully system entities and their interactions, 
while capturing notions of diversity and fairness and their inter-play with relevance.
We will seek to establish a unified framework and common methodologies for 
studying diversity and fairness in the different settings. 
As it is common in fairness ML literature, we will rely on social-justice theories, 
such as distributive justice and equitable treatment of individuals, 
as well as in social welfare objectives. 
In terms of methods, emphasis will be given to combinatorial algorithms, 
building on the previous work of the PI. 
In particular, we will consider techniques 
such as combinatorial optimization, 
optimization of submodular functions, 
local-search methods, 
greedy algorithms, 
semidefinite-programming relaxations, and convex optimization. 
Finally, we will investigate novel 
approaches based on {reinforcement learning} and {algorithms with predictions}~\cite{mitzenmacher2022algorithms}.


\vspace{-1mm}
\subsection*{4.~~~Project description}

\vspace{-1mm}
\subsubsection*{4.1.~~~Theory and methods}
\vspace{-1mm}

The project is structured along three {\em research themes}:
\vspace{-1mm}
\begin{description}
\setlength{\itemsep}{-4pt}
\item[{\exploration}\,:] 
diversity and fairness for information exploration tasks;
\item[{\networks}\,:]
diversity and fairness in information networks; and 
\item[{\markets}\,:]
diversity and fairness in information markets.
\end{description}
\vspace{-1mm}
Each theme is characterized by a distinct type of an information-access system, 
having its own unique characteristics, 
so, different abstractions and methods will be used. 
Connections among the themes~do~exist, 
% for instance, common definitions for diversity and fairness, 
and thus, we will seek to identify useful~synergies. 

\iffalse
With respect to methods, 
emphasis will be given to combinatorial algorithms,
building on the previous work of the PI 
on developing combinatorial methods for data-analysis problems.
In particular, we will consider techniques such as 
combinatorial optimization, 
optimization of submodular functions, 
local-search methods, 
greedy algorithms, 
dynamic pro\-gram\-ming, 
linear-pro\-gram\-ming and semi\-def\-ini\-te-pro\-gram\-ming relaxations, 
primal-dual methods, convex optimization,
stochastic gradient descent, etc. 
Furthermore, we will explore ideas in new domains, 
such as 
algorithms with predictions~\cite{mitzenmacher2022algorithms} and
reinforcement learning.
\fi

Next we overview the three research themes of \acronym.
\vspace{-3mm}

\subsubsection*{Research theme 1: Diversity and fairness for information-exploration tasks}
\vspace{-1mm}

\iffalse
\noindent
\hspace{-3mm}\colorbox{verylightmagenta}{
\begin{minipage}{\textwidth}
\ag{do we need a summary in a box?}
\end{minipage}}

\vspace{2mm}
\fi

We consider an information system
that stores a large collection of content items.
Users interact with the system and can access the items through various means, 
such as keyword searches, recommendations, or combinations of those.
We assume that users have different interests
and the system has prior information about user interests and user behavior.
This is a general setting that can model different scenarios, 
e.g., web-search engines, e-commerce systems, 
or platforms for browsing and reading news articles.
Our objective is to design methods that enable users to 
interact with the system and 
\emph{explore the available information}.
We are interested in systems that enable users to discover items that are relevant to their interests, 
while satisfying criteria of diverse and fair representation.

Most existing methods 
% in information retrieval and recommender systems
consider simplistic models for this problem;
for instance, \emph{find a set of $k$ items} 
that optimize some function that combines relevance with diversity. 
In the real world, however, information-exploration tasks are significantly more complex. 
First, items are presented to users in an order, and not as a set.
Second, users may click on some of the presented item, 
initiating a new ``round'' of exploration. 
Third, users may terminate their interaction with the system at any point,
based on the relevance and novelty of the presented items,
and the duration of their exploration session so far.

Motivated by these observations, 
we will introduce novel frameworks for information-exploration tasks
that model \emph{user behavior} and aim to maximize 
the total amount of knowledge accrued by the user during exploration, 
thus, combining relevance and diversity in a more intuitive way.
In our preliminary work, we have studied a simple version 
of this problem where we ask to \emph{maximize the expected diversity} of discovered content
in the presence of the probability that a user 
may terminate exploration if they do not discover interesting content.
Such a probability can be estimated from user--item relevance scores.
The problem has interesting connections to the \emph{ordered Hamiltonian-path problem}, 
and under mild assumptions we are able to design approximation algorithms
with provable guarantees.
In this project we aim to devise improved methods
and extend our framework to more sophisticated user-behavior models.
As a way to find optimal recommendation rankings in the presence of unknown user actions, 
we will also model user behavior in the setting of 
\emph{algorithms with predictions}~\cite{mitzenmacher2022algorithms}.

Another consideration in the information-exploration theme
is to consider \emph{fair representation of content} along different dimensions. 
As an example, when recommending news articles to a user, 
it is important to present more than one article from the same story,
in order to ensure \emph{coverage} of different sources with respect to their \emph{political leaning}, 
as well as with respect to their \emph{stance} towards important entities in the stories. 
% To infer entities and stances towards them we will use standard NLP toolboxes and large language models.
Additionally, we want to select stories that are well-align with user interests. 
This is relevant to the topic of \emph{calibrated recommendations}~\cite{wang2022improving}.
In \acronym, we will develop methods for recommendations that 
offer \emph{calibrated} as well as \emph{fairly-represented} content.
Furthermore, we will study problems that model the sequential-nature of recommendations~\cite{zhang2022ranking}.

Finally, we will study the information-exploration task using \emph{reinforcement learning}.
The idea would be to formulate the task as a bandit problem
where the reward depends on actions performed in earlier steps, 
quantifying diversity and fairness considerations. 
The problem can be cast as a combinatorial multi-armed bandit~\cite{chen2013combinatorial}, 
which will be the starting point for our research. 


\vspace{-1mm}
\subsubsection*{Research theme 2: Diversity and fairness in information networks}
\vspace{-1mm}

In the second research theme we will focus on information networks. 
As before, we consider a general setting, 
where networks may represent
social networks, 
hyperlink graphs, or 
``recommendation networks'' consisting of what-to-consume-next recommendations.
We view networks as systems where information spreads and content is consumed. 
Information spread and content consumption
is typically modeled via a dynamic process, such as random walks, 
shortest paths, local search, etc.

We additionally assume that network nodes are labeled according to certain attributes, 
e.g., demographic groups in a social network, or 
web page categories in a hyperlink graph. 
The confluence of network structure with node attributes typically gives rise to 
\emph{network structural bias}.
For example, contacts in a social network are more likely to have similar views on a topic, 
and thus, when navigating a network one may encounter biased information about the topic.
The problem of mitigating structural bias in information networks is an emerging topic, 
and many recent papers have addressed related questions~\cite{adriaens2022diameter,adriaens2023minimizing,cinus2023rebalancing,coupette2023reducing,fabbri2022rewiring,haddadan2022reducing}.
Typical research questions are to identify network interventions, 
such as, edge additions, re\-wirings, or re\-weightings, 
to optimize given measures of bias or polarization.

In this project, we aim to advance the state of the art in in many different directions. 
First, many of the proposed techniques consider only two groups of nodes.
Extending the existing methods for more than two groups, 
to account for realistic demographic scenarios or topics with multiple viewpoints, 
is a non trivial problem.
Second, while many of the existing works consider the problems of minimizing bias and polarization, 
the issues of diversity and fairness in information networks have not been studied.  
We propose to study a novel measure of diversity with respect to random-walk navigation, 
defined as the \emph{group cover time}, i.e., 
the expected time to visit at least one node from each group
in a random walk starting from a given node.
Our goal is to design network-intervention methods to 
optimize this diversity measure.

Additionally, while most methods consider random-walk navigation for information access, 
it is also interesting to study navigation models based on shortest-path distances. 
We recently studied the problem of minimizing the network diameter, 
while adding a small number of edges and respecting degree constraints~\cite{adriaens2022diameter}.
% As our model does not consider any node attributes, 
We aim to extend this problem in settings with node attributes. 
The goal is to identify a small number of network interventions
to optimize information access among different groups in the network.

Additionally, existing methods mainly consider \emph{global interventions}, 
i.e., they seek to optimize a global network objective. 
It will be particularly interesting to study settings for \emph{local interventions}, 
where we aim to provide \emph{recourse actions} for a particular node in the network, 
in order to optimize diversity and fairness measures for that node. 
One challenge here is to ensure that such actions do not degrade 
the corresponding measures for other nodes in the~network. 

Finally, in all of the previous cases it is assumed that interventions are certain, 
e.g., edges can be added with probability~1.
In many cases, interventions correspond to link recommendations, 
which are adopted with a certain probability
(and often the adoption probability can be estimated).
Thus, it will be interesting to study versions of those problems
in the presence of \emph{uncertainty} for the proposed interventions.

\ag{fairness-aware pagerank?}

\vspace{-1mm}
\subsubsection*{Research theme 3: Diversity and fairness in information markets}
\vspace{-1mm}

For the previous two research themes it is assumed that the content items are fixed
and notions of diversity and fairness are pertinent exclusively to the information that users receive. 
On the other hand, in many cases, modern information-access systems are \emph{two sided},
resembling \emph{information markets}, 
where the users of the system are both \emph{consumers} and \emph{producers} of content.
Examples of such systems include:
music-sharing platforms, like Spotify;
platforms used to facilitate physical-commodity markets, like renting apartments; 
platforms used for advertising and seeking jobs, like LinkedIn; and other.
%
In such an information-market system, 
consumers are interested in finding valuable content to satisfy their needs.
% whether for education, entertainment, or decision-making purposes. 
On the other side, content producers aim to attract the attention of consumers 
by creating content that is valuable to the consumers.
% For the success of such a market system, 
Both sides need to receive a sufficient amount of utility, 
% either at the form of relevant content for the consumers 
% or by the size and attention of their audience for the producers, 
otherwise they may leave the system. 

It should be clear that, in addition to the respective relevance and attention objectives, 
there are important diversity and fairness considerations at play. 
From the side of consumers, 
content diversity is important to enrich their experience and
fairness of representation is required so that they receive a well-balanced
view of the available content.
In addition, we would like to ensure that the system does not treat unfairly 
groups of users based on certain user attributes, e.g., gender or ethnicity. 
From the side of producers, 
fairness is paramount for ensuring that all producers have equal opportunities
for their content to be visible to consumers, 
instead of, say, favoring only the most popular producers, 
or members of a protected group. 

The topic of information search and content recommendations in two-sided information markets 
has recently received significant attention, 
and issues of fairness are of growing importance~\cite{do2021two,wang2021user};
one of the early papers was by the PI, 
albeit focusing on efficiency considerations~\cite{gdfm2011social}.
Despite, we believe that the area is still not well explored, 
and there are not well-established concepts and methods. 

In this project, we will seek to study novel abstractions
that capture the interplay of consumer and producer diversity and fairness, 
and complement existing measures of relevance, attention, utility, 
and welfare economics in two-sided information markets.
As in the previous research themes, 
we will incorporate models of user behavior in the system, 
study their effects to measures of interest, 
and design methods that produce recommendation sets, or ranked lists, 
that optimize such measures. 
We will also consider the fact that information markets are often hosted inside networking environments
(e.g., LinkedIn is a social network serving as a job-market platform), 
and will leverage ideas from research theme \#2 (\networks).

In addition, we will study dynamic aspects of diversity and fairness in information markets,
as consumers and producers are likely to consume and produce a series of content within several sessions.
Finally, in addition to considering the individual needs and utilities of the system actors separately, 
we will introduce notions of social welfare, 
and seek to understand the need for global social-good objectives.
For example, diversity can play a role in ensuring a healthy level of information flow in the system
and avoiding fragmentation the users into isolated communities.

\vspace{-1mm}
\subsubsection*{4.2.~~~Time plan and implementation}
\vspace{-1mm}

\instructions{
Describe summarily the time plan for the project during the grant period, 
and how the project will be implemented. 
Describe also any crucial risks or obstacles that may impact on the implementation, 
and your plan for managing these.}

\para{Time planning.}
The project will employ three doctoral students (PhD1, PhD2, PhD3) for five years each 
(working 80\% in research and 20\% as teaching assistants, as it is typical in KTH), 
and three postdoctoral researchers (PostD1, PostD2, PostD3) for two years each. 
%
% The doctoral students will be recruited on years 1, 2, and 3 of the project, 
% and the postdocs will be recruited on years 2, 3, and 4. 
% One doctoral student and one postdoctoral researcher will work on each one of the research themes,
The timeline of the project with respect to research themes and personnel is as follows:
\vspace{-2mm}
\begin{description}
\setlength{\itemsep}{-4pt}
\item[1.~~{\exploration}\,:] 
PhD1 starting on year 1, PostD1 starting on year 2;
\item[2.~~{\networks}\,:]
PhD2 starting on year 2, PostD2 starting on year 3;
\item[3.~~{\markets}\,:]
PhD3 starting on year 3, PostD3 starting on year 4;
\end{description}
\vspace{-2mm}
The rationale for \exploration starting first is to study 
fundamental concepts that can be used in the other two themes.
\markets will start last, as it has the highest level of ambition, novelty, and risk.
For each theme, the postdoc will start at the second year of the doctoral student, 
to be present and contribute in the most formative time of the student. 
%  while skipping the first year when  the students are still get adjusted and take courses.

\spara{Research output.}
We will aim publishing our work at top-tier international venues, 
focus on quality rather than quantity.
% 
% in the areas of Knowledge Discovery, Machine Learning, and Artificial Intelligence.
% We will focus on the quality of publications rather than quantity.
Target journals 
% for disseminating our work 
are IEEE TKDE, ACM TKDD, DMKD, etc. 
Target conferences are NeurIPS, ICML, VLDB, SIGMOD, SIGKDD, WebConf, WSDM, etc.

\spara{Materials.}
Our publications will be available via open access in {\small\url{arxiv.org}}. 
The software and the other outputs of the project 
will become freely available to the scientific community via {\small\url{github.com}}.

\spara{Risks and mitigation.}
Devising efficient algorithms with provable quality guarantees is a challenging task
and the largest risk of the project. 
Achieving this objective, however, offers the largest potential for scientific impact.
%  in the computer science community. 
If we are not able to prove theoretical results, 
we will study problems with simplifying assumptions, and 
will focus on devising heuristic methods and providing thorough empirical validation.

\subsubsection*{4.3.~~~Project organization}
\vspace{-1mm}

\instructions{
Clarify the contributions of yourself and any other researchers and/or key persons (including any doctoral students) 
to the implementation of the project, 
including a description of competences and roles in the project. 
Explain in particular how the time allocated by you (that is, your activity level) 
as project leader is suitable for the task, 
including the relationship with your other research undertakings.}

The PI will devote 50\% of his time in the project. 
He will supervise the doctoral students and postdocs, 
be responsible for scientific lead, 
and will allocate time to work on mathematical and algorithmic problems.
He will be responsible for ensuring flow of information and collaborations
with other groups in KTH and his international collaboration network, 
offering the possibility to the group to make research visits and internships.
The PI is currently managing an ERC Advanced Grant (2020 to 2026),
which also employs 3 doctoral students and 3 postdocs, 
so this will be a smooth transition to a similar-sized project.

% The doctoral students and postdocs will work in pairs on the corresponding research theme.
We will encourage an environment of openness and collaboration, 
while ensuring that each team member leads their own project.
%
During the hiring process we will support diversity and consider actions to achieve gender balance. 
Currently the PI supervises 6 doctoral students and the gender ratio is 3:3.

\vspace{-1mm}
\subsection*{5.~~~Need for research infrastructure}
\vspace{-1mm}

\instructions{
Specify the project's need for international and national research infrastructure. 
If you choose to use other infrastructures than those supported by the Swedish Research Council, 
and that are thereby open to all, 
you must justify this (also applies to local research infrastructure).}

The project is mainly of theoretical nature and will not require extensive computing infrastructure. 
Commodity laptops will be provided to all team members. 
For implementing and evaluating our methods we will use the available 
KTH computing facilities
and the National Academic Infrastructure for Supercomputing in Sweden (NAISS).

\vspace{-1mm}
\subsection*{6.~~~International and national collaboration}
\vspace{-1mm}

\instructions{
Describe your collaboration with foreign and Swedish researchers and research teams. 
State whether you contribute to or refer to international collaboration in your research.}

The PI has an extensive international collaboration network. 
Recent and on-going collaborations include
prof.\ De Bie in Ghent University, 
prof.\ Terzi in Bostong University,
prof.\ Mannila in Aalto University, and 
Dr.\ Bonchi in Centai Labs.
In spring 2024 the PI will spent one month as a visiting professor 
in Sapienza University of Rome, hosted by prof.\ Leonardi.
In the near future the PI will apply for a sabbatical in Stanford University, 
planning to visit prof.\ Ugander. 
We will encourage the research team to be actively involved in this collaboration network
and make research visits and internships abroad.


{\footnotesize
\setlength{\bibsep}{0pt}
\bibliographystyle{abbrv}
\bibliography{references}
}

% \newpage
% \input{rebound}

\end{document}




