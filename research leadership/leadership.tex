\documentclass[a4paper,11pt]{article}
% \usepackage{fullpage}
% \usepackage[top=2cm, bottom=2cm, left=2cm, right=2cm]{geometry}
\usepackage[margin=2cm]{geometry}
\usepackage[charter]{mathdesign}
\usepackage{epsfig}
\usepackage{graphicx}
% \usepackage{amssymb}
% \usepackage{amsmath}
\usepackage{verbatim}
\usepackage{fancyhdr}
\usepackage{pifont}% http://ctan.org/pkg/pifont
\usepackage[rgb]{xcolor}
\usepackage{tikz}
\usetikzlibrary{matrix,positioning,fit,shapes,arrows,shadows,calc,backgrounds}
\usepackage{pgfplots}
\usepackage{natbib}
\usepackage{enumitem}
\usepackage{pgfgantt}
\usepackage{booktabs}
\usepackage{url}
\usepackage{xspace}
\usepackage{wrapfig}
\usepackage[show]{chato-notes}
\usepackage{multirow}
\usepackage[official]{eurosym}
\usepackage[bottom]{footmisc}

\usepackage{lipsum}% http://ctan.org/pkg/{graphicx,lipsum}
\newcommand{\PRLsep}{\noindent\makebox[\linewidth]{\resizebox{0.3333\linewidth}{1pt}{$\bullet$}}\bigskip}

\definecolor{verylightmagenta}{rgb}{0.95,0.96,1.0}
\definecolor{brightred}{rgb}{0.90,0.2,0.2}

\newcommand{\ag}[1]{\vspace{1mm}\noindent{\color{orange}{\textbf{Comment:} #1}}}
% \newcommand{\ag}[1]{}
\newcommand{\instructions}[1]{\vspace{1mm}\noindent{\color{blue}{#1}}}
% \newcommand{\instructions}[1]{}

\newcommand{\mpara}[1]{\medskip\noindent{\bf #1}}
\newcommand{\spara}[1]{\smallskip\noindent{\bf #1}}
\newcommand{\acronym}{{\sf\small DIFIAS}\xspace}
\newcommand{\acronymtitle}{{\sf\Large DIFIAS}\xspace}

%\newcommand{\proposaltitle}{{Combating bias and polarization in online media}}
%\newcommand{\proposaltitle}{{Balancing content and reducing controversy in social media}}
\newcommand{\proposaltitle}{{Diversity and fairness in information-access systems}}
\newcommand{\proposalabstitle}{{Diversity and fairness in information-access systems}}

\newcommand{\NP}{{\ensuremath{\mathbf{NP}}}}

% CoG problems
\newcommand{\structure}{{\sc Structure}\xspace}
\newcommand{\dense}{{\sc Dense}\xspace}
\newcommand{\important}{{\sc Important}\xspace}
\newcommand{\event}{{\sc Event}\xspace}
\newcommand{\summarize}{{\sc Summarize}\xspace}

% AdG problems
\newcommand{\model}{{\sc\large Modeling}\xspace}
\newcommand{\discover}{{\sc\large Discovery}\xspace}
\newcommand{\explore}{{\sc\large Exploration}\xspace}
\newcommand{\recommend}{{\sc\large Recommendation}\xspace}
\newcommand{\ui}{{\sc\large UserInterface}\xspace}
\newcommand{\methods}{{\sc\large methods}\xspace}

\newcommand{\rto}{{\bf RT1}\xspace}
\newcommand{\rtw}{{\bf RT2}\xspace}
\newcommand{\rtr}{{\bf RT3}\xspace}

%% squishlist
\newcommand{\squishlist}{\begin{list}{$\bullet$}
  { \setlength{\itemsep}{-1pt}
     \setlength{\parsep}{2pt}
     \setlength{\topsep}{2pt}
     \setlength{\partopsep}{0pt}
     \setlength{\leftmargin}{1.5em}
     \setlength{\labelwidth}{1em}
     \setlength{\labelsep}{0.5em} } }
\newcommand{\squishend}{
\end{list}  }


%% gantt stuff

% \definecolor{foobarblue}{RGB}{0,153,255}
% \definecolor{foobaryellow}{RGB}{234,187,0}
% \newganttchartelement{foobar}{
%     foobar/.style={
%         shape=rounded rectangle,
%         inner sep=0pt,
%         draw=foobarblue!50!black,
%         very thick,
%         top color=white,
%         bottom color=foobarblue!50
%     },
%     foobar incomplete/.style={
%         /pgfgantt/foobar,
%         draw=foobaryellow,
%         bottom color=foobaryellow!50
%     },
%     foobar label font=\slshape,
%     foobar left shift=-.1,
%     foobar right shift=.1
% }

%% tikz stuff

\pgfdeclarelayer{background}
\pgfdeclarelayer{foreground}
\pgfsetlayers{background,main,foreground}


\makeatletter
\tikzset{multicircle/.style  args={#1, #2}{%
 alias=tmp@name, % 
  postaction={%
    insert path={
     \pgfextra{% 
     \pgfpointdiff{\pgfpointanchor{\pgf@node@name}{center}}%
                  {\pgfpointanchor{\pgf@node@name}{east}}%            
     \pgfmathsetmacro\insiderad{\pgf@x}%
     %\foreach \c [count=\ci from = 0, evaluate=\ci as \angle using 360 - (\ci) * #1] in {#2}%
        \fill[white] (\pgf@node@name.center)  circle (\insiderad-\pgflinewidth);%
        \draw[#2] (\pgf@node@name.center)  circle (\insiderad-\pgflinewidth);%
        \fill[#2] (\pgf@node@name.center)  -- ++(0:\insiderad-\pgflinewidth) arc (0:#1:\insiderad-\pgflinewidth)--cycle;%
        }}}}}
\makeatother

\definecolor{yafaxiscolor}{rgb}{0.3, 0.3, 0.3}
\definecolor{yafcolor1}{rgb}{0.4, 0.165, 0.553}
\definecolor{yafcolor2}{rgb}{0.949, 0.482, 0.216}
\definecolor{yafcolor3}{rgb}{0.47, 0.549, 0.306}
\definecolor{yafcolor4}{rgb}{0.925, 0.165, 0.224}
\definecolor{yafcolor5}{rgb}{0.141, 0.345, 0.643}
\definecolor{yafcolor6}{rgb}{0.965, 0.933, 0.267}
\definecolor{yafcolor7}{rgb}{0.627, 0.118, 0.165}
\definecolor{yafcolor8}{rgb}{0.878, 0.475, 0.686}
\definecolor{yafcolor9}{rgb}{0.965, 0.733, 0.767}

\newlength{\yafaxispad}
\setlength{\yafaxispad}{-4pt}
\newlength{\yaftlpad}
\setlength{\yaftlpad}{\yafaxispad}
\addtolength{\yaftlpad}{-0pt}
\newlength{\yaflabelpad}
\setlength{\yaflabelpad}{-2pt}
\newlength{\yafaxiswidth}
\setlength{\yafaxiswidth}{1.2pt}
\newlength{\yafticklen}
\setlength{\yafticklen}{2pt}

\makeatletter
\def\pgfplots@drawtickgridlines@INSTALLCLIP@onorientedsurf#1{}
\makeatother

\newcommand{\yafdrawxaxis}[2]{
	\pgfplotstransformcoordinatex{#1}\let\xmincoord=\pgfmathresult 
	\pgfplotstransformcoordinatex{#2}\let\xmaxcoord=\pgfmathresult 
	\pgfsetlinewidth{\yafaxiswidth} 
	\pgfsetcolor{yafaxiscolor}
	\pgfpathmoveto{\pgfpointadd{\pgfpointadd{\pgfplotspointrelaxisxy{0}{0}}{\pgfqpointxy{\xmincoord}{0}}}{\pgfqpoint{-0.5\yafaxiswidth}{\yafaxispad}}}
	\pgfpathlineto{\pgfpointadd{\pgfpointadd{\pgfplotspointrelaxisxy{0}{0}}{\pgfqpointxy{\xmaxcoord}{0}}}{\pgfqpoint{0.5\yafaxiswidth}{\yafaxispad}}}
	\pgfusepath{stroke}

}
\newcommand{\yafdrawyaxis}[2]{
	\pgfplotstransformcoordinatey{#1}\let\ymincoord=\pgfmathresult 
	\pgfplotstransformcoordinatey{#2}\let\ymaxcoord=\pgfmathresult 
	\pgfsetlinewidth{\yafaxiswidth} 
	\pgfsetcolor{yafaxiscolor}
	\pgfpathmoveto{\pgfpointadd{\pgfpointadd{\pgfplotspointrelaxisxy{0}{0}}{\pgfqpointxy{0}{\ymincoord}}}{\pgfqpoint{\yafaxispad}{-0.5\yafaxiswidth}}}
	\pgfpathlineto{\pgfpointadd{\pgfpointadd{\pgfplotspointrelaxisxy{0}{0}}{\pgfqpointxy{0}{\ymaxcoord}}}{\pgfqpoint{\yafaxispad}{0.5\yafaxiswidth}}}
	\pgfusepath{stroke}
}

\newcommand{\yafdrawaxis}[4]{\yafdrawxaxis{#1}{#2}\yafdrawyaxis{#3}{#4}}

\pgfplotscreateplotcyclelist{yaf}{% 
{yafcolor1,mark options={scale=0.75},mark=o}, 
{yafcolor2,mark options={scale=0.75},mark=square},
{yafcolor3,mark options={scale=0.75},mark=triangle},
{yafcolor4,mark options={scale=0.75},mark=o},
{yafcolor5,mark options={scale=0.75},mark=o},
{yafcolor6,mark options={scale=0.75},mark=o},
{yafcolor7,mark options={scale=0.75},mark=o},
{yafcolor8,mark options={scale=0.75},mark=o}} 

\pgfplotsset{axis y line=left, axis x line=bottom,
	tick align=outside,
	compat = 1.3,
	tickwidth=\yafticklen,
	clip = false,
	every axis title shift = 0pt,
    x axis line style= {-, line width = 0pt, opacity = 0},
    y axis line style= {-, line width = 0pt, opacity = 0},
    x tick style= {line width = \yafaxiswidth, color=yafaxiscolor, yshift = \yafaxispad},
    y tick style= {line width = \yafaxiswidth, color=yafaxiscolor, xshift = \yafaxispad},
    x tick label style = {font=\scriptsize, yshift = \yaftlpad},
    y tick label style = {font=\scriptsize, xshift = \yaftlpad},
    every axis y label/.style = {at = {(ticklabel cs:0.5)}, rotate=90, anchor=center, font=\scriptsize, yshift = -\yaflabelpad},
    every axis x label/.style = {at = {(ticklabel cs:0.5)}, anchor=center, font=\scriptsize, yshift = \yaflabelpad},
    x tick label style = {font=\scriptsize, yshift = 1pt},
    grid = major,
    major grid style  = {dash pattern = on 1pt off 3 pt},
	every axis plot post/.append style= {line width=\yafaxiswidth} ,
	legend cell align = left,
	legend style = {inner sep = 1pt, cells = {font=\scriptsize}},
	legend image code/.code={%
		\draw[mark repeat=2,mark phase=2,#1] 
		plot coordinates { (0cm,0cm) (0.15cm,0cm) (0.3cm,0cm) };% 
	} 
}



% \setcounter{page}{1}

\renewcommand{\baselinestretch}{1.05} 
\begin{document}


\begin{center} 
% {\large Vetenskapsrådet: Distinguished professor grant within natural and engineering sciences 2024} \vspace{2.5mm}\\
{\Large Research leadership} \vspace{3mm}\\
{\Large\bf {\proposaltitle} {\sc (}{\acronymtitle}{\sc )}}  \vspace{3mm} \\
{\Large Aristides Gionis} 
\end{center}


\subsection*{1.~~~Research topics and track record}

My work on data science spans a career of over 20 years, 
which includes numerous contributions in algorithmic data-analysis research, 
and consolidates experience from academia and industry.
I have been in academia since 2013 
(2013 to 2019 in Aalto University and 2020 to present in KTH).
Prior to that I was a senior research scientist in Yahoo!\ Research Labs. 

The main focus of my work has been in data mining and algorithmic data analysis. 
My approach and philosophy to research is driven by considering practical problems 
motivated by real-world application scenarios, 
and then abstracting the problems to simpler mathematical formulations 
and developing novel and theoretically-rigorous solutions.

Over the years I have published over 60 journal and over 170 conference publications, 
in peer-reviewed top-tier international journals and conferences.
Many of my publications have attracted significant attention in the community
and have generated a volume of follow-up works.
The areas that I have made significant contributions and 
have published highly-cited papers include the following: 
\squishlist
\item[--~] similarity search in high dimensional spaces; 
\item[--~] significance testing for data mining results; 
\item[--~] graph mining: efficient computation of graph motifs, discovery of dense subgraphs, community detection, role mining, network inference, etc.;
\item[--~] social-network and social-media analysis;
\item[--~] analysis of temporal networks and information-diffusion in networks;
\item[--~] analysis of urban data;
\item[--~] data clustering;
\item[--~] interpretable models for machine learning.
\squishend

My publications have received \emph{1 test-of-time award}, 
\emph{4 best-paper awards}, 
and \emph{4 best-student paper awards}, co-authored by students under my supervision. 


\subsection*{2.~~~Research supervision}

I am proud of the doctoral students and postdocs that I have supervised over the years. 
I want to believe that I have inspired them by being example for hard work,
hands-on style of research, 
high standards of quality and academic integrity, 
and always treating them as peers. \\
In Aalto I have supervised (as main supervisor) 10 doctoral students and 8 postdocs, all graduated. \\
In KTH I have supervised (as main supervisor) 1 doctoral student and 2 postdocs, all graduated. \\
Currently, in KTH I am supervising (as main supervisor) 6 doctoral students and 5 postdocs. \\
Some examples of the current positions of the doctoral students and postdocs that I have supervised are the following:
\squishlist
\item[--~] Eric Malmi, PhD student, 2018,\footnote{The year graduation year.} Google.
\item[--~] Sanja Scepanovic, PhD student, 2018, Bell Labs.
\item[--~] Kiran Garimella, PhD student, 2018, assistant professor in Rutgers University.
\item[--~] Polina Rozenshtein, PhD student, 2019, Amazon.
\item[--~] Suhas Thejaswi, PhD student, 2022, postdoc in MPI.
\item[--~] Nikolaj Tatti, postdoc, 2017, associate professor in the University of Helsinki.
\item[--~] Michael Mathioudakis,  postdoc, 2017, associate professor in the University of Helsinki.
\item[--~] Stefan Neumann, postdoc, 2023, assistant professor in TU Vienna.
\squishend

\subsection*{3.~~~International recognition}

My recognition from the international research community is reflected 
in my participation in the editorial board of two high-impact journals: 
ACM Transaction on the Web (TWEB), and Data Mining and Knowledge Discovery (DMKD). 
I have consistently participated as a senior PC member or area chair in 
most major conferences in data discovery and knowledge management in the last 10, or more, years.
Additionally, I have been a program co-chair of the following conferences:
ECML PKDD 2010, ACM WSDM 2013, the Web Conference 2022, and ACM WSDM 2024. 

I have been invited for seminars and keynote presentations in many occasions, 
including 
the European Conference on Machine Learning and Data Mining, 
the Conference on Complex Networks, 
the Austrian Computer Science Day, 
the distinguished lecture series in EPFL, 
and numerous international workshops and symposiums.

Finally, I have been an examiner in 30 PhD dissertation committees.


\end{document}




